\newcommand{\institut}{Institut f\"ur Telekommunikationssysteme}
\newcommand{\fachgebiet}{Nachrichten\"ubertragung}
\newcommand{\veranstaltung}{Praktikum Nachrichten\"ubertragung}
\newcommand{\pdfautor}{Dirk Babendererde (321 836), Thomas Kapa (325 219)}
\newcommand{\autor}{Dirk Babendererde (321 836)\\ Thomas Kapa (325 219)}
\newcommand{\gruppe}{Gruppe:}
\newcommand{\betreuer}{Betreuer: Lieven Lange}


\newcommand{\pdftitle}{Nachrichtenuebertragung\ Praktikum\ 06}
\newcommand{\prototitle}{Praktikum 06 \\ Digitale Übertragungstechnik: Digitale Empf\"anger}

\input{../../packages/tu_header_9}

% damit das outline funktioniert noch mal:
\begin{document}


%     \lstinputlisting{./praktikum6.sce}

%---------------------------------------------------------------------
%---------------------------------------------------------------------
%---------------------------------------------------------------------

\section{Vorbereitung}
\begin{quote}
    
    
    
\end{quote}


\section{Labordurchführung}
\begin{quote}
    
    
    \subsection{Encoderkennlinie}
    \begin{quote}
        
        Um die Bitfehlerrate messen zu können, werden die D/A-Box, das PCM-DECODER-Modul, die ADDER-Module, Das Picoscope
       und der NOISE GENERATOR benötigt.\\
       Der oberste Ausgang (rot) der D/A-Box, der das Datensignal enthält, wird auf das erste ADDER-Modul gegeben, dass
       keine Verstärkungsmöglichkeit enthält und mit einem Offset aus dem VARIABLE DC-Modul addiert. Anschließend wird
       das Ergebnis auf das zweite ADDER-Modul geführt, wo es mit dem Verstärkungsregler auf eine Amlitude von -1..1
       gedämpft wird.\\
       Der 3. Ausgabe von unten der D/A-Box (gelb) enthält die PCM codierten Datenworte. Diese werden auf den Eingang
       PCM DATA des PCM DECODER-Moduls gegeben.\\
       Der 2. Ausgang von unten enthält das Rahmensignal, welches mit dem Gegenstück des PCM-DECODER-Moduls verbunden
       wird.\\
       Der unterste Ausgang (blau) gibt das Clock Signal aus. Dieses wird mit einem T-Stück zum einen an den Eingang B
       des Picoscope und zum anderen an den CLK Eingang des PCM DECODER-Moduls des Picoscope geschlossen.\\
       Zuletzt wird der OUTPUT des PCM_Decoder Moduls (Spannung des Faktors zur Verstärkung bzw. Dämpfung des Rauschens)
       auf einen Multiplizierer gegeben und mit -6 dB multipliziert. Der Ausgang des Multiplizierers wird auf den
       zweiten Eingang des zweiten ADDER-Moduls gegeben.\\
       Der Ausgang dieses ADDER-Moduls wird auf den A Eingang des Picoscopes gegeben.
        
    \end{quote}
    
   
\end{quote}

\section{Auswertung \& Theorie}
\begin{quote}
    
    \subsection{Encoderkennlinie}
    
    \begin{quote}
        
    
    
    \end{quote}
    
    
    \subsection{Quantisierungfehler}
    
    \begin{quote}
        
    \end{quote}    
    
\end{quote}

\section{Zusammenfassung}
\begin{quote}
	
 
    
\end{quote} % end Sec: section


\end{document}