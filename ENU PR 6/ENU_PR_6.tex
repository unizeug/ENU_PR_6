\newcommand{\institut}{Institut f\"ur Telekommunikationssysteme}
\newcommand{\fachgebiet}{Nachrichten\"ubertragung}
\newcommand{\veranstaltung}{Praktikum Nachrichten\"ubertragung}
\newcommand{\pdfautor}{Dirk Babendererde (321 836), Thomas Kapa (325 219)}
\newcommand{\autor}{Dirk Babendererde (321 836)\\ Thomas Kapa (325 219)}
\newcommand{\gruppe}{Gruppe:}
\newcommand{\betreuer}{Betreuer: Lieven Lange}


\newcommand{\pdftitle}{Nachrichtenuebertragung\ Praktikum\ 06}
\newcommand{\prototitle}{Praktikum 06 \\ Digitale Übertragungstechnik: Digitale Empf\"anger}

\input{../../packages/tu_header_9}

% damit das outline funktioniert noch mal:
\begin{document}


%     \lstinputlisting{./praktikum6.sce}

%---------------------------------------------------------------------
%---------------------------------------------------------------------
%---------------------------------------------------------------------

\section{Vorbereitung}
\begin{quote}
       Die Ergebnisse der Simulation der Wasserfallkurven ist in Abb. xxx zu sehen. 
  
    \begin{equation}
	     \begin{split}
		\rho_{01}=\frac{1}{E_{b}} \int\limits_{0}^{T_{Bit}}  s_{0}(t) \cdot s_{1}(t)  \ dt
		\footnote[1]{Prof. Dr.-Ing. Sikora, Thomas, Foliensatz 10 Binäre Basisbandübertragung, Einführung in die
           Nachrichtenübertragung, S.129}
	     \end{split}
    \end{equation} 
     
    
    \begin{equation}
	     \begin{split}
		SNR_{E}=\frac{4E_{b}}{N_{0}}(1-\rho_{01})
		\footnote[1]{Prof. Dr.-Ing. Sikora, Thomas, Foliensatz 10 Binäre Basisbandübertragung, Einführung in die
           Nachrichtenübertragung, S.129}
		 \label{eq:SNR}
	     \end{split}
    \end{equation}  
    
    Theoretisch würde man also wegen des Faktors $1-\rho_{01}$ für $\rho_{01}=0$ in Formel \ref{eq:SNR} das kleinste SNR
    und damit den höchsten Bitfehler erwarten. Dies ist in Abb. xxx aber kaum zu sehen. $\rho_{01}=0$ und
    $\rho_{01}=-1/3$ liegen nahezu aufeinander. Dies erklärt sich durch die höhere Bitenergie. Da für $\rho_{01}=0$ für
    die Sendeform 4 Baud (4 Baud heißt in diesem Fall ein Bit besteht aus vier Zeichen, entweder -1 oder 1) benutzt werden, 
    ergibt sich nach Formel \ref{eq:E_b} ein $E_{b}$ von 4 statt von 3 für $\rho_{01} = -1/3$ und $\rho_{01}= -1$. Das
    $E_{b}$ verbessert wiederum das SNR in Formel \ref{eq:SNR} und damit verringert es auch die
    Bitfehlerwahrscheinlichkeit.
    
    \begin{equation}
	     \begin{split}
		   E_{b}= \int\limits_{-\infty}^{\infty}  \abs{s(t)}^2 \ dt
		   \footnote[1]{Prof. Dr.-Ing. Sikora, Thomas, Foliensatz 10 Binäre Basisbandübertragung, Einführung in die
           Nachrichtenübertragung, S.129}
		 \label{eq:E_b}
	     \end{split}
    \end{equation}  
    
    
    \begin{equation}
	     \begin{split}
		p_{Bit}=\frac{1}{2}erfc  \sqrt{\frac{E_{b}}{2N_{0}(1-\rho_{01})}}
		\footnote[1]{Prof. Dr.-Ing. Sikora, Thomas, Foliensatz 10 Binäre Basisbandübertragung, Einführung in die
           Nachrichtenübertragung, S.129}
		 \label{eq:p_bit}
	     \end{split}
    \end{equation}  

An Formel \ref{eq:p_bit} kann man erkennen, dass die Sendeform, also ob Rechteck, Dreieck, oder Cosinus keinen Einfluss
auf die Bifehlerwahrscheinlichkeit hat.
    
    
\end{quote}


\section{Labordurchführung}
\begin{quote}
    
    
    \subsection{Encoderkennlinie}
    \begin{quote}
        
        Um die Bitfehlerrate messen zu können, werden die D/A-Box, das PCM-DECODER-Modul, die ADDER-Module, Das Picoscope
       und der NOISE GENERATOR benötigt.\\
       Der oberste Ausgang (rot) der D/A-Box, der das Datensignal enthält, wird auf das erste ADDER-Modul gegeben, dass
       keine Verstärkungsmöglichkeit enthält und mit einem Offset aus dem VARIABLE DC-Modul addiert. Anschließend wird
       das Ergebnis auf das zweite ADDER-Modul geführt, wo es mit dem Verstärkungsregler auf eine Amlitude von -1..1
       gedämpft wird.\\
       Der 3. Ausgabe von unten der D/A-Box (gelb) enthält die PCM codierten Datenworte. Diese werden auf den Eingang
       PCM DATA des PCM DECODER-Moduls gegeben.\\
       Der 2. Ausgang von unten enthält das Rahmensignal, welches mit dem Gegenstück des PCM-DECODER-Moduls verbunden
       wird.\\
       Der unterste Ausgang (blau) gibt das Clock Signal aus. Dieses wird mit einem T-Stück zum einen an den Eingang B
       des Picoscope und zum anderen an den CLK Eingang des PCM DECODER-Moduls des Picoscope geschlossen.\\
       Zuletzt wird der OUTPUT des PCM_Decoder Moduls (Spannung des Faktors zur Verstärkung bzw. Dämpfung des Rauschens)
       auf einen Multiplizierer gegeben und mit -6 dB multipliziert. Der Ausgang des Multiplizierers wird auf den
       zweiten Eingang des zweiten ADDER-Moduls gegeben.\\
       Der Ausgang dieses ADDER-Moduls wird auf den A Eingang des Picoscopes gegeben.
        
    \end{quote}
    
   
\end{quote}

\section{Auswertung \& Theorie}
\begin{quote}
    
    \subsection{Encoderkennlinie}
    
    \begin{quote}
        
    
    
    \end{quote}
    
    
    \subsection{Quantisierungfehler}
    
    \begin{quote}
        
    \end{quote}    
    
\end{quote}

\section{Zusammenfassung}
\begin{quote}
	
 
    
\end{quote} % end Sec: section


\end{document}